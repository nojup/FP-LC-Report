\documentclass[12pt,a4paper]{article}
\input{latexmacros.tex}

\title{My Report}
\author{Me}
\date{Autumn 2016}
\hypersetup{pdfauthor={Me}, pdftitle={My Report}}

\begin{document}

\maketitle

\begin{abstract}
We give a toy example of a report in \emph{literate programming} style.
The main advantage of this is that source code and documentation can
be written and presented next to each other.
We use the listings package to typeset Haskell source code nicely.
\end{abstract}

\vfill

\tableofcontents

\clearpage

% We include one file for each section. The ones containing code should
% be called something.lhs and also mentioned in the .cabal file.


\section{How to use this?}

To generate the PDF, open the report.tex in your favorite LaTeX editor and
hit compile, or manually do this:

\begin{verbatim}
  pdflatex report
  bibtex report
  pdflatex report
  pdflatex report
\end{verbatim}

You should have stack installed (see \url{http://haskellstack.org/}) and
open a terminal in the same folder.

\begin{itemize}
  \item To compile everything: \verb|stack build|.
  \item To open ghci and play with your code: \verb|stack ghci|
  \item To run the executable from Section \ref{sec:Main}: \verb|stack build && stack exec myprogram|
  \item To run the tests from Section \ref{sec:simpletests}: \verb|stack clean && stack test --coverage|
\end{itemize}


\input{Basics.lhs}

\input{Main.lhs}

\input{Conclusion.tex}

\addcontentsline{toc}{section}{Bibliography}
\bibliographystyle{alpha}
\bibliography{references.bib}

\end{document}
